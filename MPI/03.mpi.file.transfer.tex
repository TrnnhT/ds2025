\documentclass{article}
\usepackage{graphicx}
\usepackage{listings}
\usepackage{amsmath}

\title{Practical work 3}
\author{}
\date{}

\begin{document}
\maketitle


\section{Why MPI Implementation?}
We used MPI (via mpi4py) because it is a robust and widely used framework for parallel programming. Its communication primitives (send, receive,...) simplify implementation for distributed tasks like file transfer.

\section{System Design}
The system adopts a master-worker architecture:
\begin{itemize}
    \item \textbf{Client:} Reads the file and divides it into chunks.
    \item \textbf{Host:} Receive file chunks from the master and save them locally.
\end{itemize}

\section{System Organization}
Each MPI process is assigned a rank:
\begin{itemize}
    \item Rank 0 (Host): Manages the file and sends chunks.
    \item Other Ranks (Client): Receive and save file chunks.
\end{itemize}


\section{Code Implementation}
The implementation utilizes mpi4py for communication. Below are snippets demonstrating key functions:

\subsection{Host}
\begin{lstlisting}[language=Python]
if rank == 0:
    with open(filename, 'rb') as file:
        file_data = file.read()
        for i in range(1, size):
            comm.send(file_chunk, dest=i, tag=0)
\end{lstlisting}

\subsection{Client}
\begin{lstlisting}[language=Python]
if rank != 0:
    data = comm.recv(source=0, tag=0)
    with open(f'worker_{rank}_output.txt', 'wb') as file:
        file.write(data)
\end{lstlisting}

\section{Who does what}
- Host and Client adjust : Tran Hong Nhat
\end{document}
