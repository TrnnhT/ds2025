\documentclass{article}
\usepackage{amsmath}
\usepackage{graphicx}
\usepackage{listings}

\title{File Transfer System: Client-Server Architecture}
\author{}
\date{\today}

\begin{document}

\maketitle

\section{Client Code}
The client connects to the server and receives a file in chunks. The key steps are as follows:

\begin{itemize}
    \item The client creates a socket, connects to the server at IP address \texttt{127.0.0.1} on port \texttt{1001}, and prepares to receive a file.
    \item The file is received in chunks of 1024 bytes using the \texttt{recv} function and saved locally as \texttt{file.txt}.
    \item If there is no data received, the connection is closed, and the file is considered received.
\end{itemize}

The relevant part of the client code that handles file reception is:

\begin{lstlisting}[language=Python]
    client_socket.connect((ip_host, port))
    with open(filename, 'wb') as transfer_file:
        while True:
            content = client_socket.recv(1024)
            if not content:
                break  
            transfer_file.write(content)  
\end{lstlisting}

\section{Server Code}
The server listens for incoming connections and sends a file to the client. The steps involved are:

\begin{itemize}
    \item The server creates a socket, binds it to the IP address \texttt{127.0.0.1} and port \texttt{1001}, and waits for a client to connect.
    \item Once a connection is established, the server reads the file \texttt{file.txt} in chunks of 1024 bytes and sends it to the client.
    \item If the file is not found or there is an error, the server sends an error message to the client.
\end{itemize}

The relevant part of the server code that handles file sending is:

\begin{lstlisting}[language=Python]
    connect, address = server_socket.accept()
    with open(filename, 'rb') as file:
        while True:
            data = file.read(1024)
            if not data:
                break  
            connect.send(data)
\end{lstlisting}

\section{File Transfer Process}
The file transfer involves the following steps:

\begin{itemize}
    \item The client connects to the server via TCP at IP \texttt{127.0.0.1} and port \texttt{1001}.
    \item The server accepts the connection, opens the file, and sends it in 1024-byte chunks.
    \item The client receives the chunks, writes them to a file, and completes the transfer when all data is received.
\end{itemize}

The file transfer ends when the client receives all the data and closes the connection.

\section{Conclusion}
The client-server system successfully transfers a file from the server to the client using Python’s socket library. The transfer happens in chunks over a TCP connection, ensuring reliable delivery of data. Possible improvements include error handling for network issues, larger file support, and encryption for secure transmission.

\end{document}
