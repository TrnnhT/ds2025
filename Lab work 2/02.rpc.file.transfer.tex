\documentclass[12pt]{article}
\usepackage{graphicx}
\usepackage{geometry}
\geometry{a4paper, margin=1in}

\title{Practical work 2}

\begin{document}

\maketitle

\section*{RPC Service Design}
The RPC service is implemented using Python's built-in \texttt{xmlrpc} library. The server hosts a function, \texttt{send\_file()}, which reads a file in binary mode and sends its content to the client.


\section*{System Organization}
The system consists of two components:
\begin{enumerate}
    \item \textbf{Server:} Hosts the RPC function for file transfer.
    \item \textbf{Client:} Requests the file and saves it locally.
\end{enumerate}


\section*{Implementation}
\subsection*{Code Snippet: Server}
\begin{verbatim}
from xmlrpc.server import SimpleXMLRPCServer
import os

def send_file():
    filename = "file.txt"
    if os.path.exists(filename):
        with open(filename, 'rb') as file:
            return file.read()
    else:
        return "Error"

server = SimpleXMLRPCServer(("127.0.0.1", 7090))
server.register_function(send_file, "send_file")
server.serve_forever()
\end{verbatim}

\subsection*{Code Snippet: Client}
\begin{verbatim}
import xmlrpc.client

client = xmlrpc.client.ServerProxy("http://127.0.0.1:7090")
file_content = client.send_file()
with open("received_file.txt", "wb") as file:
    file.write(file_content)
\end{verbatim}

\section*{Who does what}
\begin{itemize}
    \item Tran Hong Nhat: ADjust the client and server code file
\end{itemize}

\end{document}
